\documentclass[UTF8]{ctexart}
% \documentclass{article}
\usepackage{geometry, CJKutf8}
\geometry{margin=1.5cm, vmargin={0pt,1cm}}
\setlength{\topmargin}{-1cm}
\setlength{\paperheight}{29.7cm}
\setlength{\textheight}{25.3cm}

% useful packages.
\usepackage{amsfonts}
\usepackage{amsmath}
\usepackage{amssymb}
\usepackage{amsthm}
\usepackage{enumerate}
\usepackage{graphicx}
\usepackage{multicol}
\usepackage{fancyhdr}
\usepackage{layout}
\usepackage{listings}
\usepackage{float, caption}
\usepackage{graphicx}
\usepackage{subfigure}


\lstset{
    basicstyle=\ttfamily, basewidth=0.5em
}

% some common command
\newcommand{\dif}{\mathrm{d}}
\newcommand{\avg}[1]{\left\langle #1 \right\rangle}
\newcommand{\difFrac}[2]{\frac{\dif #1}{\dif #2}}
\newcommand{\pdfFrac}[2]{\frac{\partial #1}{\partial #2}}
\newcommand{\OFL}{\mathrm{OFL}}
\newcommand{\UFL}{\mathrm{UFL}}
\newcommand{\fl}{\mathrm{fl}}
\newcommand{\op}{\odot}
\newcommand{\Eabs}{E_{\mathrm{abs}}}
\newcommand{\Erel}{E_{\mathrm{rel}}}

\begin{document}

\pagestyle{fancy}
\fancyhead{}
\lhead{马浩天, 3240102534}
\chead{数据结构与算法第七次作业}
\rhead{Dec 2nd, 2024}

\section{HeapSort 的写法}
采取经典的利用堆排序进行原地排序的思路:在原 vector 先建成一个大根堆,然后依次取出堆顶并维护堆的形态。

具体的实现方法在 HeapSort.h 中可见,对于类 HeapSort,我写了如下成员函数:
在 private 中: buildHeap 建堆操作, percloateDown 降堆操作;
在 public 中:sort 表示堆排序。

调用的时候只需要写 HeapSort<Comparable>::sort(array) 即可,其中 array 的类型是 vector<Comparable>。


\section {测试流程}

我的测试范围包括五种序列:

1. 完全随机序列(Random Array)
2. 有序序列(Ordered Array)
3. 逆序序列(Reversed Array)
4. 单一重复序列(Single-Repetitive Array,单个元素重复出现70\% 及以上的数列)
5. 多重重复序列(Multi-Repetitive Array, 出现次数多于1的元素的总数多于70\% 的数列)

造数据方面,生成随机数使用 std::mt19937,计时使用了 std::chrono ,单位为 microseconds (1 microsecond = 1e-9 second),序列长度取了 $3 \times 10^6$。

生成重复序列的方法是:shuffle 一个下标序列获得一个置换,从而能够让重复的元素均匀分配在所有的位置上。

为了减少误差,对于每一种序列进行了五次重复实验。编译选项里面打开了 -O2 ,否则 std 速度会非常缓慢。

同时也使用了 valgrind 进行内存泄漏测试,没有发生任何内存泄漏。

\section {测试结果}


\begin{table}[h]
    \begin{tabular}{|l|c|c|}
    \hline
    Sequence Type & \multicolumn{1}{l|}{Custom Heapsort TimeCost(microseconds)} & \multicolumn{1}{l|}{std::sort\_heap TimeCost(microseconds)} \\ \hline
    Ramdom Array            & 250873 & 246003 \\ \hline
    Ordered Array           & 103165 & 108229 \\ \hline
    Reversed Array          & 112118 & 118599 \\ \hline
    Single-Repetitive Array & 106240 & 133842 \\ \hline
    Multi-Repetitive Array  & 143304 & 155408 \\ \hline
    \end{tabular}
    \caption{在 5 次重复实验下的测试结果}
    \label{tab:my-table}
    \end{table}
    
    
    
\section {对于测试结果的分析}
    
    可以注意到,Custom HeapSort 实际上几乎和 std::sort\_heap 的效率相当,而且有可能更快,这可能与标准库进行了过多的封装,为了通用的设计牺牲了一些效率有关,也可能与 Cache Miss 有关。
    
\end{document}
%%% Local Variables: 
%%% mode: latex
%%% TeX-master: t
%%% End: 
