\documentclass[UTF8]{ctexart}
% \documentclass{article}
\usepackage{geometry, CJKutf8}
\geometry{margin=1.5cm, vmargin={0pt,1cm}}
\setlength{\topmargin}{-1cm}
\setlength{\paperheight}{29.7cm}
\setlength{\textheight}{25.3cm}

% useful packages.
\usepackage{amsfonts}
\usepackage{amsmath}
\usepackage{amssymb}
\usepackage{amsthm}
\usepackage{enumerate}
\usepackage{graphicx}
\usepackage{multicol}
\usepackage{fancyhdr}
\usepackage{layout}
\usepackage{listings}
\usepackage{float, caption}
\usepackage{graphicx}
\usepackage{subfigure}


\lstset{
    basicstyle=\ttfamily, basewidth=0.5em
}

% some common command
\newcommand{\dif}{\mathrm{d}}
\newcommand{\avg}[1]{\left\langle #1 \right\rangle}
\newcommand{\difFrac}[2]{\frac{\dif #1}{\dif #2}}
\newcommand{\pdfFrac}[2]{\frac{\partial #1}{\partial #2}}
\newcommand{\OFL}{\mathrm{OFL}}
\newcommand{\UFL}{\mathrm{UFL}}
\newcommand{\fl}{\mathrm{fl}}
\newcommand{\op}{\odot}
\newcommand{\Eabs}{E_{\mathrm{abs}}}
\newcommand{\Erel}{E_{\mathrm{rel}}}

\begin{document}

\pagestyle{fancy}
\fancyhead{}
\lhead{马浩天, 3240102534}
\chead{数据结构与算法项目作业-表达式求值}
\rhead{Dec 15th, 2024}

\section{项目介绍}

    本项目是浙江大学《数据结构与算法》课程的项目作业,设计并完成了对于一系列中缀表达式的求值程序,支持多重括号和四则运算,支持有限位小数运算,支持负数输入(更进一步地,处理了所有可能的正负号,一个数字前面可以有多个正负号),支持科学计数法输入。
    能够初步识别各种非法表达式,包括括号失配(\texttt{bracketMismatchException}),运算符失配(\texttt{operatorMismatchException}),除 $0$ 错误(\texttt{dividedByZeroException}),非法字符(\texttt{invalidCharacterException}),非法数字(\texttt{invalidInputNumberException}),表达式过长(\texttt{expressionSizeOutOfBoundException})等。
    对输入的表达式有长度要求。

\section{设计思路}

程序的设计主要分为两个模块:数字的读入和表达式求值。

\subsection{数字的读入}
    一个可能包含科学计数法的数字应当包括如下三部分:整数部分、小数部分、指数部分,分隔符分别是 \texttt{.} 和 \textbf{e}。程序定位并找出了每一部分的数据,并且对于不合法的情况丢出 \texttt{invalidInputNumberException} 异常。

\subsection{表达式求值}
    使用 \texttt{std::stack} 来辅助求值,包括一个运算符栈和一个数字栈。首先判断是否有非法字符。然后,自左向右逐个扫描表达式的元素,如果是左括号,将其压入运算符栈;如果是右括号,那么取出栈顶的运算符并进行运算,直至栈顶出现左括号;如果是运算符,那么先把运算符栈顶处所有优先级更高的运算符取出并求值,然后再将当前运算符压入栈中;如果是数字,那么调用数字输入模块来读取,并压入数字栈中。
    注意如下要点:

        1. 一元运算符(正负号)的处理:首先,一元运算符只可能在左括号或者运算符的后一位出现,其余的一律认为是加减号;其次,一元运算符的优先级应当是最高的,程序中进行了特殊处理。

        2. 在取出运算符进行运算的时候,注意判断除以 0 的错误。

        3. 在扫描完毕过后栈中可能还有未处理完的运算符。

        4. 如果处理右括号的时候栈顶不出现左括号,或者左括号在最后被当作运算符处理,说明出现了括号失配。

        5. 如果在处理运算符的时候发现数字栈已空,或者在所有运算符处理完毕后数字栈里仍有多于一个元素,说明出现了运算符失配。

\section{测试数据和结果}


\begin{table}[!ht]
    \centering
    \begin{tabular}{|l|l|}
    \hline
        \textbf{Expression(s)} & \textbf{Result(s)} \\ \hline
        \texttt{3+5} & \texttt{8} \\ \hline
        \texttt{6-2} & \texttt{4} \\ \hline
        \texttt{7-9} & \texttt{-2} \\ \hline
        \texttt{4*8} & \texttt{32} \\ \hline
        \texttt{7/3} & \texttt{2.33333} \\ \hline
    \end{tabular}
    \caption{基础四则运算}
\end{table}

\begin{table}[!ht]
    \centering
    \begin{tabular}{|l|l|}
    \hline
        \textbf{Expression(s)} & \textbf{Result(s)} \\ \hline
        \texttt{+5} & \texttt{5 }\\ \hline
        \texttt{-3} & \texttt{-3 }\\ \hline
        \texttt{++3} & \texttt{3 }\\ \hline
        \texttt{--5} & \texttt{5 }\\ \hline
        \texttt{-----5} & \texttt{-5 }\\ \hline
        \texttt{-+-+--++-2} & \texttt{-2 }\\ \hline
        \texttt{++-+-+--+4} & \texttt{4 }\\ \hline
    \end{tabular}
    \caption{一元运算符处理}
\end{table}

\begin{table}[!ht]
    \centering
    \begin{tabular}{|l|l|}
    \hline
        \textbf{Expression(s)} & \textbf{Result(s)} \\ \hline
        \texttt{1.13} & \texttt{1.13} \\ \hline
        \texttt{1e5} & \texttt{100000} \\ \hline
        \texttt{1.5e2} & \texttt{150} \\ \hline
        \texttt{2e-3} & \texttt{0.002} \\ \hline
        \texttt{-5.2e-1} & \texttt{-0.52} \\ \hline
    \end{tabular}
    \caption{小数和科学计数法处理}
\end{table}

\begin{table}[!ht]
    \centering
    \begin{tabular}{|l|l|}
    \hline
        \textbf{Expression(s)} & \textbf{Result(s)} \\ \hline
        \texttt{(3+5)*4} & \texttt{32} \\ \hline
        \texttt{(2*(3+5))-4} & \texttt{12} \\ \hline
        \texttt{((1+3)+(4*-4))/(2)} & \texttt{-6} \\ \hline
    \end{tabular}
    \caption{有括号参与的运算}
\end{table}

\begin{table}[!ht]
    \centering
    \begin{tabular}{|l|l|}
    \hline
        \textbf{Expression(s)} & \textbf{Result(s)} \\ \hline
        \texttt{5+8*9-4} & \texttt{73} \\ \hline
        \texttt{2/3+6} & \texttt{6.66667} \\ \hline
        \texttt{7+7*7/-7} & \texttt{0} \\ \hline
        \texttt{-4*6} & \texttt{-24} \\ \hline
    \end{tabular}
    \caption{纯四则运算}
\end{table}
\begin{table}[!ht]
    \centering
    \begin{tabular}{|l|l|}
    \hline
        \textbf{Expression(s)} & \textbf{Result(s)} \\ \hline
        \texttt{-3+5*2-(1.5e3/5e2)} & \texttt{4} \\ \hline
        \texttt{(---5.0*-(1e-3+1e-2-1e-1*1/1.2e0))*1.5} & \texttt{-0.5425} \\ \hline
        \texttt{(-+9.4e2--6.2*(1.14+5.14))/(----2e1)} & \texttt{-45.0532} \\ \hline
        \texttt{((1e3-500)/2) + 3*(1.5e2)} & \texttt{700} \\ \hline
    \end{tabular}
    \caption{复杂表达式}
\end{table}
\begin{table}[!ht]
    \centering
    \begin{tabular}{|l|l|}
    \hline
        \textbf{Expression(s)} & \textbf{Result(s)} \\ \hline
        $\underbrace{1+1+\cdots+1+1}_{10000\text{个}1}$ & \texttt{5000} \\ \hline
        $\underbrace{((\cdots(1+1)\cdots))}_{4000\text{个}()}$ & \texttt{2} \\ \hline
    \end{tabular}
    \caption{极限情况}
\end{table}
\begin{table}[!ht]
    \centering
    \begin{tabular}{|c|}
    \hline
    \multicolumn{1}{|l|}{\textbf{Invalid Expression(s)}} \\ \hline
    \texttt{1e}                             \\ \hline
    \texttt{.2}                             \\ \hline
    \texttt{3+1e5.1}                                     \\ \hline
    \texttt{1e2e3/4}                                     \\ \hline
    \texttt{122.122.122.122}                             \\ \hline
    \texttt{1.e3}                                        \\ \hline
    \texttt{1e.6*8}                                      \\ \hline
    \end{tabular}
    \caption{Exception Case 1 : \texttt{invalidInputNumberException}}
    % \label{tab:my-table}
\end{table}
\begin{table}[!ht]
    \centering
    \begin{tabular}{|c|}
    \hline
    \multicolumn{1}{|l|}{Invalid Expression(s)} \\ \hline
    \texttt{6/0}                                     \\ \hline
    \texttt{4*2/(2e1-6.0*3-2)}                       \\ \hline
    \end{tabular}
    \caption{Exception Case 2 : \texttt{dividedByZeroException}}
    % \label{tab:my-table}
\end{table}

\begin{table}[!ht]
    \centering
    \begin{tabular}{|c|}
    \hline
    \multicolumn{1}{|l|}{Invalid Expression(s)} \\ \hline
    \texttt{(((5+6)*7)/6))}                              \\ \hline
    \texttt{)9*8)}                                       \\ \hline
    \texttt{22-((4)}                                     \\ \hline
    \end{tabular}
    \caption{Exception Case 3 : \texttt{bracketMismatchException}}
    % \label{tab:my-table}
\end{table}


\begin{table}[!ht]
    \centering
    \begin{tabular}{|c|}
    \hline
    \multicolumn{1}{|l|}{Invalid Expression(s)} \\ \hline
    \texttt{1+}                              \\ \hline
    \texttt{() - 9}                                       \\ \hline
    \texttt{5**8}                                       \\ \hline
    \texttt{6(3)}                                     \\ \hline
    \end{tabular}
    \caption{Exception Case 4 : \texttt{operatorMismatchException}}
    % \label{tab:my-table}
\end{table}


\begin{table}[!ht]
    \centering
    \begin{tabular}{|c|}
    \hline
    \multicolumn{1}{|l|}{Invalid Expression(s)} \\ \hline
    \texttt{as6d54w8}                              \\ \hline
    \texttt{1e2*3ll}                                     \\ \hline
    \texttt{HappyNewYear!}                                       \\ \hline
    \texttt{WhyNotBlackMyth:Wukong?}                                     \\ \hline
    \end{tabular}
    \caption{Exception Case 5 : \texttt{invalidCharacterException}}
    % \label{tab:my-table}
\end{table}


\end{document}
%%% Local Variables: 
%%% mode: latex
%%% TeX-master: t
%%% End: 
