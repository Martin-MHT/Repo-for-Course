\documentclass[UTF8]{ctexart}
% \documentclass{article}
\usepackage{geometry, CJKutf8}
\geometry{margin=1.5cm, vmargin={0pt,1cm}}
\setlength{\topmargin}{-1cm}
\setlength{\paperheight}{29.7cm}
\setlength{\textheight}{25.3cm}

% useful packages.
\usepackage{amsfonts}
\usepackage{amsmath}
\usepackage{amssymb}
\usepackage{amsthm}
\usepackage{enumerate}
\usepackage{graphicx}
\usepackage{multicol}
\usepackage{fancyhdr}
\usepackage{layout}
\usepackage{listings}
\usepackage{float, caption}
\usepackage{graphicx}
\usepackage{subfigure}


\lstset{
    basicstyle=\ttfamily, basewidth=0.5em
}

% some common command
\newcommand{\dif}{\mathrm{d}}
\newcommand{\avg}[1]{\left\langle #1 \right\rangle}
\newcommand{\difFrac}[2]{\frac{\dif #1}{\dif #2}}
\newcommand{\pdfFrac}[2]{\frac{\partial #1}{\partial #2}}
\newcommand{\OFL}{\mathrm{OFL}}
\newcommand{\UFL}{\mathrm{UFL}}
\newcommand{\fl}{\mathrm{fl}}
\newcommand{\op}{\odot}
\newcommand{\Eabs}{E_{\mathrm{abs}}}
\newcommand{\Erel}{E_{\mathrm{rel}}}

\begin{document}

\pagestyle{fancy}
\fancyhead{}
\lhead{马浩天, 3240102534}
\chead{数据结构与算法第六次作业}
\rhead{Oct.16th, 2024}

\section{remove 函数的实现}
原有的 remove 函数在需要删除当前节点,且左右子树都存在的情况下采取如下操作:在右子树中寻找最小的元素,赋给当前节点,然后递归地删除右子树中最小的节点。事实上,这个过程只会发生一次,因此在树高合理的情况下,仅仅考虑树的结构的话,时间复杂度可以接受。唯一的问题在于\textbf{赋值}操作时间复杂度的不可控性。

为了改进这个问题,我在不改变原有删除思路的条件下,优化了赋值操作。首先,我写了 detachMin 方法,作用是:“查找子树中最小元素,删除所在节点,并返回节点的指针。”随后,在 remove 函数中我先调用 detachMin 方法寻找并删除右子树中最小节点,将返回的节点的左子树和右子树分别指向欲删除节点的左右子树,这样就可以直接将当前指针指向返回的节点。最后删除了返回的指针,这样就避免了 element 的复制,也不会产生内存泄漏。

在平衡方法上,采用 AVL 树的平衡手段,通过旋转操作使左右子树的高度差不超过1。
只要是可能影响树的形态的操作(如 insert, remove, detachMin 等等),都要在返回之前进行平衡操作。
我将平衡操作打包成了 balance 函数,每次旋转过后更新最新的高度。
注意:不要忘记在 detachMin 中平衡树高。


\end{document}

%%% Local Variables: 
%%% mode: latex
%%% TeX-master: t
%%% End: 
