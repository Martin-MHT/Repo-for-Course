\documentclass[UTF8]{ctexart}
\usepackage{geometry, CJKutf8}
\geometry{margin=1.5cm, vmargin={0pt,1cm}}
\setlength{\topmargin}{-1cm}
\setlength{\paperheight}{29.7cm}
\setlength{\textheight}{25.3cm}

% useful packages.
\usepackage{amsfonts}
\usepackage{amsmath}
\usepackage{amssymb}
\usepackage{amsthm}
\usepackage{enumerate}
\usepackage{graphicx}
\usepackage{multicol}
\usepackage{fancyhdr}
\usepackage{layout}
\usepackage{listings}
\usepackage{float, caption}

\lstset{
    basicstyle=\ttfamily, basewidth=0.5em
}

% some common command
\newcommand{\dif}{\mathrm{d}}
\newcommand{\avg}[1]{\left\langle #1 \right\rangle}
\newcommand{\difFrac}[2]{\frac{\dif #1}{\dif #2}}
\newcommand{\pdfFrac}[2]{\frac{\partial #1}{\partial #2}}
\newcommand{\OFL}{\mathrm{OFL}}
\newcommand{\UFL}{\mathrm{UFL}}
\newcommand{\fl}{\mathrm{fl}}
\newcommand{\op}{\odot}
\newcommand{\Eabs}{E_{\mathrm{abs}}}
\newcommand{\Erel}{E_{\mathrm{rel}}}

\begin{document}

\pagestyle{fancy}
\fancyhead{}
\lhead{马浩天, 3240102534}
\chead{数据结构与算法第四次作业}
\rhead{Oct.16th, 2024}

\section{测试程序的设计思路和预期输出}

我的测试程序分为 10 个部分。

\subsection{copy}

拷贝测试验证了如下内容:拷贝构造函数、赋值语句、自我赋值、=运算符的重载、深度拷贝。
我创建了 1 个原始列表 l1 和 3 个用于验证如上内容的列表 l2, l3, l4。l2 验证了 = 运算符的有效重载,l3 验证了拷贝构造函数,l4 则验证了连续赋值和自我赋值是否处理得当。
在将它们全部输出完毕后,我对 l2, l3 分别进行了修改,然后再次将它们全部输出来验证深度拷贝。

\subsection{push\_back}
我创建了一个空列表,随后调用 push\_back() 方法依次插入了 1, 2, 3 三个元素,最后输出了整个列表的内容。按照预期,程序应当输出 1 2 3。
\subsection{push\_front}
我创建了一个空列表,随后调用 push\_back() 方法依次插入了 1, 2, 3 三个元素,最后输出了整个列表的内容。按照预期,程序应当输出 3 2 1。
\subsection{pop\_back}
我创建了一个列表并初始化为\{1, 2, 3\},之后调用 pop\_back 方法,最后输出了整个列表的内容。按照预期,程序应当输出 1 2。
\subsection{pop\_front}
我创建了一个列表并初始化为\{1, 2, 3\},之后调用 pop\_front 方法,最后输出了整个列表的内容。按照预期,程序应当输出 2 3。
\subsection{iterator}
对迭代器的测试包括前置/后置++运算和前置/后置--运算。
我创建了一个列表并初始化为 \{1, 2, 3, 4, 5\}。随后,利用迭代器定位了尾部元素 5,并按顺序出了后置/前置--、后置/前置++后迭代器所指的元素。随后,我连续调用了三次前置--和两次前置++。按照预期,程序应当输出 5 3 3 5 2 4。
\subsection{insert}
我创建了一个列表,并初始化为 \{1, 3\},之后调用迭代器定位了 3,并调用 insert 方法在它之前插入了 2,最后输出了整个列表的内容和返回的迭代器所指的元素。按照预期,程序应当输出 1 2 3 \textbackslash n2。
\subsection{erase\_single}
我创建了一个列表并初始化为 \{1, 2, 3\},之后调用迭代器定位了 2,并调用 erase 方法删除了它,最后输出了整个列表的内容和返回的迭代器所指的元素。按照预期,程序应当输出 1 3 \textbackslash n3。
\subsection{erase\_range}
我创建了一个列表并初始化为 \{1, 2, 3, 4, 5\},之后用迭代器和自增操作将区间起点和中点分别定位在了 2,4。随后,调用了 erase 方法删除了这个区间内的元素。最后输出了整个列表的内容和返回的迭代器所指的元素。由于区间是左闭右开的,因此程序按照预期应当输出 1 4 5 \textbackslash n4。
\subsection{size and empty}
我创建了一个空链表并测试了 size 和 empty 方法的返回值。随后,分别调用了 push\_back、push\_front、pop\_front、pop\_back、erase(single)、erase(range) 所有可能的操作并检查每次操作后的 size 和 empty。最后,我还检查了通过 initializer\_list 方法初始化的链表的 size 和 empty。


\section{测试的结果}

程序按照预期完成了所有的操作,所有的输出都与预期相同。

我用 valgrind 进行测试,没有发生内存泄露。

\section{对于区间删除的性能优化}

在下发的代码中,提示可以优化区间删除的实现。先前的实现方法是通过单点删除的方法依次删去 [head, tail) 的所有元素,然而,这其中有许多不必要的指针修改。实际上,除了区间起点和终点处及附近,其他位置的指针修改都是不必要的,毕竟它们终究会被删除。根据这个思路,我重构了区间删除的实现。具体地,直接将头的 next 指向尾部,尾的 prev 指针指向头部,中间删除时不再修改任何指针。注意左闭右开以及更新 size。

\end{document}

%%% Local Variables: 
%%% mode: latex
%%% TeX-master: t
%%% End: 
